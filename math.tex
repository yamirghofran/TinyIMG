\documentclass{article}
\usepackage{amsmath}

\title{Image Editor Project Math}
\author{}
\date{}

\begin{document}

\maketitle

\section{Rotating an Image (by \(\theta\) degrees)}
Rotation moves the image around a point (usually the center). The rotation matrix is:

\[
R(\theta) = \begin{bmatrix}
\cos(\theta) & -\sin(\theta) \\
\sin(\theta) & \cos(\theta)
\end{bmatrix}
\]

\paragraph{Example:} If \(\theta = 90^\circ\), then \(\cos(90^\circ) = 0\), \(\sin(90^\circ) = 1\), so the matrix is:

\[
R(90^\circ) = \begin{bmatrix}
0 & -1 \\
1 & 0
\end{bmatrix}
\]

This swaps and negates coordinates to rotate the image.

\section{Scaling (Resizing the Image)}
Scaling makes an image larger or smaller. The scaling matrix is:

\[
S = \begin{bmatrix}
s_x & 0 \\
0 & s_y
\end{bmatrix}
\]

\begin{itemize}
    \item \(s_x\) = scaling factor in the x-direction
    \item \(s_y\) = scaling factor in the y-direction
\end{itemize}

\paragraph{Example:} If you double the size, use \(s_x = 2, s_y = 2\).

\paragraph{Example:} If you shrink by half, use \(s_x = 0.5, s_y = 0.5\).

\section{Stretching (Uneven Scaling)}
Stretching is similar to scaling, but differently in each direction.

\begin{itemize}
    \item Stretch horizontally: \(s_x > 1, s_y = 1\)
    \item Stretch vertically: \(s_x = 1, s_y > 1\)
\end{itemize}

\section{Cropping (Not a Matrix Operation)}
Cropping is just cutting out a part of an image. You select a region and discard the rest—this doesn’t need a matrix.

\section{Flipping (Mirroring)}
Flipping means reversing the image horizontally or vertically.

\begin{itemize}
    \item Horizontal Flip (Mirror left-right):
    \[
    \begin{bmatrix}
    -1 & 0 \\
    0 & 1
    \end{bmatrix}
    \]
    \item Vertical Flip (Mirror top-bottom):
    \[
    \begin{bmatrix}
    1 & 0 \\
    0 & -1
    \end{bmatrix}
    \]
\end{itemize}

\section{Warp (Advanced Transformation)}
Warping means distorting the image, like perspective shifts. A simple shear matrix is:

\[
W = \begin{bmatrix}
1 & k_x \\
k_y & 1
\end{bmatrix}
\]

\begin{itemize}
    \item \(k_x\) controls horizontal slant
    \item \(k_y\) controls vertical slant
\end{itemize}

\paragraph{Example:} If \(k_x = 1, k_y = 0\), the image leans right.

\section{Basic Color Filters (Advanced)}
Color changes don’t affect positions, but each pixel’s RGB values are multiplied by a matrix. For grayscale conversion, each pixel’s (R, G, B) values are transformed to a single grayscale value.

\section{Identity Matrix}
For 2D transformations, it's:

\[
I = \begin{bmatrix}
1 & 0 \\
0 & 1
\end{bmatrix}
\]

For colors (3D transformations), it's:

\[
I_3 = \begin{bmatrix}
1 & 0 & 0 \\
0 & 1 & 0 \\
0 & 0 & 1
\end{bmatrix}
\]

\subsection{Why Do We Need It?}
\begin{enumerate}
    \item \textbf{Starting Point:} Before applying any transformations (rotation, scaling, flipping, etc.), we start with something neutral. The identity matrix is a good default because it does nothing to the image.
    \item \textbf{Combining Multiple Transformations:} If we want to apply multiple transformations at once, we multiply matrices together. Starting with the identity matrix makes this process structured. 

    \paragraph{Example:} If you want to scale and then rotate, you multiply:
    \[
    \text{Final Matrix} = R \times S \times I
    \]

    \item \textbf{Preserves Original Data (Prevents Accidental Modifications):} When coding transformations, it's better to start with \(I\) and apply changes step by step. This prevents unexpected distortions or errors.
\end{enumerate}

\paragraph{Example With Identity Matrix}
\begin{enumerate}
    \item Start with \(I\) (original state).
    \item Multiply by scaling matrix.
    \item Multiply by rotation matrix.
    \item The final matrix gives the correct transformation.
\end{enumerate}

\end{document}
